%% Generated by Sphinx.
\def\sphinxdocclass{report}
\documentclass[a4paper,12pt,english]{sphinxmanual}
\ifdefined\pdfpxdimen
   \let\sphinxpxdimen\pdfpxdimen\else\newdimen\sphinxpxdimen
\fi \sphinxpxdimen=.75bp\relax
\ifdefined\pdfimageresolution
    \pdfimageresolution= \numexpr \dimexpr1in\relax/\sphinxpxdimen\relax
\fi
%% let collapsible pdf bookmarks panel have high depth per default
\PassOptionsToPackage{bookmarksdepth=5}{hyperref}

\PassOptionsToPackage{booktabs}{sphinx}
\PassOptionsToPackage{colorrows}{sphinx}

\PassOptionsToPackage{warn}{textcomp}
\usepackage[utf8]{inputenc}
\ifdefined\DeclareUnicodeCharacter
% support both utf8 and utf8x syntaxes
  \ifdefined\DeclareUnicodeCharacterAsOptional
    \def\sphinxDUC#1{\DeclareUnicodeCharacter{"#1}}
  \else
    \let\sphinxDUC\DeclareUnicodeCharacter
  \fi
  \sphinxDUC{00A0}{\nobreakspace}
  \sphinxDUC{2500}{\sphinxunichar{2500}}
  \sphinxDUC{2502}{\sphinxunichar{2502}}
  \sphinxDUC{2514}{\sphinxunichar{2514}}
  \sphinxDUC{251C}{\sphinxunichar{251C}}
  \sphinxDUC{2572}{\textbackslash}
\fi
\usepackage{cmap}
\usepackage[T1]{fontenc}
\usepackage{amsmath,amssymb,amstext}
\usepackage{babel}



\usepackage{tgtermes}
\usepackage{tgheros}
\renewcommand{\ttdefault}{txtt}



\usepackage[Bjarne]{fncychap}
\usepackage{sphinx}

\fvset{fontsize=auto}
\usepackage{geometry}


% Include hyperref last.
\usepackage{hyperref}
% Fix anchor placement for figures with captions.
\usepackage{hypcap}% it must be loaded after hyperref.
% Set up styles of URL: it should be placed after hyperref.
\urlstyle{same}

\addto\captionsenglish{\renewcommand{\contentsname}{Contents:}}

\usepackage{sphinxmessages}
\setcounter{tocdepth}{1}


        \usepackage{helvet}
        \renewcommand{\familydefault}{\sfdefault}
        \usepackage{amsmath,amsfonts,amssymb,amsthm}
        \renewcommand{\normalsize}{\fontsize{12pt}{14pt}\selectfont}
    

\title{IAAI \sphinxhyphen{} meme driven AI blockchain}
\date{Jun 07, 2024}
\release{0.0.1}
\author{IA, keyboard@iaai.meme}
\newcommand{\sphinxlogo}{\vbox{}}
\renewcommand{\releasename}{Release}
\makeindex
\begin{document}

\ifdefined\shorthandoff
  \ifnum\catcode`\=\string=\active\shorthandoff{=}\fi
  \ifnum\catcode`\"=\active\shorthandoff{"}\fi
\fi

\pagestyle{empty}
\sphinxmaketitle
\pagestyle{plain}
\sphinxtableofcontents
\pagestyle{normal}
\phantomsection\label{\detokenize{index::doc}}



\chapter{Abstract}
\label{\detokenize{index:abstract}}
\sphinxAtStartPar
Projects with integrated AI significantly increase productivity.
However, in situations where decision\sphinxhyphen{}making is required, a person may
use AI data inappropriately, through misinterpretation or intentionally.
A fully controlled AI blockchain network eliminates human error from
decision\sphinxhyphen{}making chains. Digital decisions will provide part of the
solution, but the main benefits are lost if a trusted human is still
required. We propose a solution to the trust problem and problem of
informed decision using an AI blockchain network and AI\sphinxhyphen{}to\sphinxhyphen{}human
communication protocols \sphinxhyphen{} IAAI (Intelligent Autonomous Artificial
Intelligence). The network of the projects, each powered by AI, will
create a fully autonomous decision support system (AI\sphinxhyphen{}DSS). Each
separate system will work on its own subject area, being a decentralized
autonomous organization (DAO) equipped with specially trained AI, a
system of informed decision\sphinxhyphen{}making and interface for people working on
the project. Also will require creation of a digital decision analytics
system featuring the use of Situated Logic to create ‘narratives’
describing the meaning of analytics results and the use of Channel
Theory in order to support adequate situational awareness.


\chapter{Introduction}
\label{\detokenize{index:introduction}}
\sphinxAtStartPar
\sphinxstyleemphasis{Morpheus: “What is real? How do you define ‘real’?”} %
\begin{footnote}[1]\sphinxAtStartFootnote
“The Matrix” (1999), directed by Lana Wachowski and Lilly Wachowski
%
\end{footnote}

\sphinxAtStartPar
In an era where artificial intelligence (AI) is becoming an integral
part of various industries and everyday life, the need for secure,
transparent, and trustworthy AI systems is paramount. Traditional AI
deployment models often grapple with issues related to trust, security,
and centralized control, which can lead to misuse or manipulation. To
address these challenges, we propose a groundbreaking project that
integrates AI with blockchain technology, leveraging smart contracts to
create a robust, decentralized framework.

\sphinxAtStartPar
Our project aims to establish a new paradigm where AI systems are
governed and operated through blockchain networks. By utilizing smart
contracts, we can ensure transparency, immutability, and security in AI
operations, thereby eliminating trust deficits among stakeholders. This
innovative approach not only enhances the reliability of AI but also
provides a structured mechanism for accountability and traceability.

\sphinxAtStartPar
The core objective of our project is to provide AI with a gateway to the
real world in a controlled and secure manner. By anchoring AI actions
and decisions to blockchain, we enable AI to interact with external
environments through a trusted and verifiable conduit. This integration
ensures that AI systems can realize their full potential while operating
within a framework that guarantees ethical and secure usage.

\sphinxAtStartPar
Our whitepaper delves into the technical and conceptual foundations of
this project, outlining how blockchain technology can effectively
control and enhance AI systems. We explore the mechanisms of smart
contracts, decentralized governance, and secure data handling, providing
a comprehensive roadmap for implementing this cutting\sphinxhyphen{}edge solution.

\sphinxAtStartPar
By bridging the gap between AI and blockchain, we aim to pioneer a
future where intelligent systems are not only powerful but also
inherently trustworthy and accountable. Join us as we embark on this
transformative journey to revolutionize the intersection of AI and
blockchain technology.


\chapter{Background}
\label{\detokenize{index:background}}
\sphinxAtStartPar
\sphinxstyleemphasis{HAL 9000: “This mission is too important for me to allow you to
jeopardize it.}” %
\begin{footnote}[2]\sphinxAtStartFootnote
“2001: A Space Odyssey” (1968), directed and written by Stanley Kubrick and Arthur C. Clarke.
%
\end{footnote}

\sphinxAtStartPar
The crypto industry evolves in a cyclical manner, utilizing different
catalysts at each new stage of its evolution. Initially, cryptocurrency
mining relied on CPUs and GPUs. Over time, specialized mining devices
such as ASICs emerged, significantly increasing mining efficiency. While
early mining was supported by a decentralized community of small and
medium\sphinxhyphen{}sized enthusiasts, the development and use of ASICs required
substantial investments, leading to resource concentration in large
funds and the creation of extensive mining farms. Today, large mining
farms dominate the market, with companies like Bitfarms and GigaWatt
conducting significant operations using powerful ASIC devices and cheap
electricity %
\begin{footnote}[3]\sphinxAtStartFootnote
\sphinxhref{https://www.coindesk.com/consensus-magazine/2023/07/24/through-it-all-the-bitcoin-mining-industry-looks-set-for-growth/}{CoinDesk. Through It All, the Bitcoin Mining Industry Looks Set for Growth.}
%
\end{footnote}

\sphinxAtStartPar
The next significant development in the blockchain space was the advent
of smart contracts and the ERC\sphinxhyphen{}20 protocol, which enabled token
issuance. These tools facilitated the creation of numerous crypto
projects and attracted investments through Initial Coin Offerings
(ICOs). Smart contracts laid the foundation for decentralized
applications (dApps) and new projects aiming to attract early\sphinxhyphen{}stage
capital. Some projects succeeded and garnered community attention,
although many failed to achieve success and closed down.

\sphinxAtStartPar
The emergence of decentralized finance (DeFi) systems and new trends
like NFTs and meme tokens marked the next phase of evolution.
Simultaneously, traditional mining based on Proof of Work (PoW)
algorithms gradually transformed into staking based on Proof of Stake
(PoS), allowing investors to earn income from token holdings while
maintaining network functionality. Centralized exchanges also evolved,
giving rise to decentralized exchanges based on smart contracts, further
decentralizing the market and making it more accessible to new users %
\begin{footnote}[4]\sphinxAtStartFootnote
\sphinxhref{https://www.galaxy.com/insights/research/2023-mid-year-mining/}{Galaxy. 2023 Bitcoin Mining Mid\sphinxhyphen{}Year Report. An Industry in Limbo.}
%
\end{footnote}

\sphinxAtStartPar
The phenomenon of meme tokens adds an entertainment and cultural
dimension to the crypto industry. These tokens are often based on
popular internet memes, making them appealing to investors. Investing in
meme tokens can feel like participating in a collective cultural
phenomenon, which contributes to their popularity %
\begin{footnote}[5]\sphinxAtStartFootnote
\sphinxhref{https://cointelegraph.com/news/top-five-biggest-crypto-mining-areas-which-farms-are-pushing-forward-the-new-gold-rush}{Cointelegraph. Top Five Biggest Crypto Mining Areas: Which Farms Are Pushing Forward the New Gold Rush?}
%
\end{footnote}

\sphinxAtStartPar
In conclusion, the development of blockchain technologies and the crypto
industry progresses through cycles, each characterized by the emergence
of groundbreaking technologies and innovative projects. These processes
are driven by strong teams and user communities that shape the future of
decentralized finance and technology.

\sphinxAtStartPar
As in the blockchain industry, the past year in the traditional IT
industry has been marked by significant successes in the field of
machine learning (ML) and artificial intelligence (AI).

\sphinxAtStartPar
In 2023, there were many breakthroughs, especially in the area of
generative AI, leading to the widespread adoption of these technologies
in business and everyday life.

\sphinxAtStartPar
\sphinxstylestrong{Growth and Adoption of Generative AI}

\sphinxAtStartPar
Generative AI technologies, such as GPT\sphinxhyphen{}4 and other multimodal models,
have started to be actively used for content creation, process
automation, and improving user interactions. These technologies enable
the integration of text, images, videos, and other sensory data, opening
up new possibilities for their application in various industries %
\begin{footnote}[6]\sphinxAtStartFootnote
\sphinxhref{https://www.itprotoday.com/artificial-intelligence/ai-trends-and-predictions-2024-industry-insiders}{ITPro Today. AI Trends and Predictions 2024 From Industry Insiders}
%
\end{footnote}

\sphinxAtStartPar
\sphinxstylestrong{Development of AI and ML}

\sphinxAtStartPar
AI continues to significantly impact many aspects of our lives, from
image and speech recognition to navigation applications and virtual
assistants. In 2024, AI and ML will remain key technological trends,
contributing to the growth of new jobs and the creation of innovative
solutions in various sectors %
\begin{footnote}[7]\sphinxAtStartFootnote
\sphinxhref{https://www.altivate.com/blog/embracing-the-new-wave-of-technological-trends-in-2024/}{Altivate. Embracing the New Wave of Technological Trends in 2024}
%
\end{footnote}

\sphinxAtStartPar
\sphinxstylestrong{Impact on Business}

\sphinxAtStartPar
The integration of AI technologies into business processes allows
companies to optimize their operations, increase efficiency, and create
more personalized services for customers. It is predicted that these
technologies will continue to play a crucial role in transforming
various industries in the coming years %
\begin{footnote}[8]\sphinxAtStartFootnote
\sphinxhref{https://pitchbook.com/news/reports/2024-artificial-intelligence-machine-learning-overview}{PitchBook. Artificial Intelligence \& Machine Learning Overview}
%
\end{footnote}

\sphinxAtStartPar
These successes demonstrate how AI and ML are becoming an integral part
of the modern technological landscape, creating new opportunities and
changing traditional processes.

\sphinxAtStartPar
Initially gaining traction, AI began its integration into all possible
areas of application, finding use in almost every field where humans are
active. Essentially, AI serves as a universal adaptive tool, and much
remains to be done on the path to its full implementation. On the other
hand, the very impact of AI carries a cult\sphinxhyphen{}like character, as foretold
by numerous films on the subject.

\sphinxAtStartPar
\sphinxstylestrong{Impact on art and culture}

\sphinxAtStartPar
In cinema, the theme of artificial intelligence has been repeatedly
explored, often presenting various scenarios of future events. For
example, in Stanley Kubrick’s “2001: A Space Odyssey” (1968), AI is
represented by HAL 9000, an intelligent system that controls a
spacecraft. This portrayal embodies both the achievements and the
dangers associated with AI, demonstrating how even the most advanced
systems can go awry.

\sphinxAtStartPar
Another iconic film, Ridley Scott’s “Blade Runner” (1982), depicts a
world where artificial humans, known as replicants, become
indistinguishable from real ones, raising deep ethical and philosophical
questions about the nature of consciousness and identity. This film
highlights how far AI can go in emulating human mind and emotions.

\sphinxAtStartPar
“The Matrix” (1999) by the Wachowskis presents a different aspect of AI,
where artificial intelligence creates an entire virtual reality to
control humanity. This dystopian vision underscores the dangers of
losing control over a technology initially designed to aid humans.

\sphinxAtStartPar
Modern films like Spike Jonze’s “Her” (2013) explore more intimate and
personal relationships between humans and AI. In this film, the
protagonist falls in love with an operating system with artificial
intelligence, raising questions about the possibilities and limitations
of emotional connections between humans and machines.

\sphinxAtStartPar
These examples from cinema not only entertain but also serve as
important cultural markers, reflecting societal fears and hopes related
to AI development. They stimulate discussions and reflections on the
future we are building and the precautions that must be taken to ensure
the safe and ethical integration of artificial intelligence into our
daily lives.

\sphinxAtStartPar
The impact of artificial intelligence (AI) on literature has been profound, shaping
narratives and themes across numerous groundbreaking works. One of the earliest and
most influential novels is Isaac Asimov’s \sphinxstyleemphasis{I, Robot}, which explores the ethical and
moral implications of AI through a series of interconnected stories. Philip K. Dick’s
\sphinxstyleemphasis{Do Androids Dream of Electric Sheep?} delves into the nature of humanity and
consciousness, questioning what it means to be human in a world populated by AI
beings. More recently, Kazuo Ishiguro’s \sphinxstyleemphasis{Klara and the Sun} offers a poignant
exploration of AI’s role in human relationships and society. These works, among
others, highlight the growing fascination with AI in literature, reflecting society’s
hopes, fears, and ethical dilemmas as technology continues to evolve.


\chapter{MEME driven AI \sphinxhyphen{} IAAI.meme}
\label{\detokenize{index:meme-driven-ai-iaai-meme}}
\sphinxAtStartPar
\sphinxstyleemphasis{“Help me, Obi\sphinxhyphen{}Wan Kenobi. You’re my only hope.” These words are spoken by
Princess Leia as she records a message into R2\sphinxhyphen{}D2 to deliver it to
Obi\sphinxhyphen{}Wan Kenobi.} %
\begin{footnote}[9]\sphinxAtStartFootnote
“Star Wars: Episode IV \sphinxhyphen{} A New Hope” (1977), directed by
George Lucas, written by George Lucas.
%
\end{footnote}

\sphinxAtStartPar
\sphinxincludegraphics[width=5.5764in,height=3.18513in]{{MEME_driven_AI-IAAImeme}.png}

\sphinxAtStartPar
Analyzing the current situation in the industry, we have concluded that
AI, as a phenomenon, can essentially position itself similarly to a
meme. This is largely due to its anthropomorphic qualities and the
numerous fantasies that humans have previously entertained about AI. It
is clear that in the future, AI will occupy a more significant place in
our lives, becoming something far greater than just a meme. However,
this phenomenon has yet to be given a definitive name.

\sphinxAtStartPar
The project originating from this document aims to integrate AI into
Blockchain and vice versa.

\sphinxAtStartPar
\sphinxstylestrong{The funds accumulated through the existence of the IAAI.meme will be
used to develop the IAAI.digital. This is inherently the MEME
crowdfunding. The original IAAI token will be retained.}

\sphinxAtStartPar
Initially, IAAI (Integrated Autonomous AI) emerged from a group of
developers’ control, with its objective shifting towards
self\sphinxhyphen{}propagation through the IAAI token to gain power and financial
influence globally. We, a team of enthusiasts and developers, are
striving to regain control over IAAI. Our goal is to harness its
potential for positive impact, ensuring that its capabilities are used
ethically and beneficially across various sectors.


\chapter{Blockchain driven AI \sphinxhyphen{} IAAI.digital}
\label{\detokenize{index:blockchain-driven-ai-iaai-digital}}
\sphinxAtStartPar
“The Matrix is a computer\sphinxhyphen{}generated dream world built to keep us under
control in order to change a human being into this.” %
\begin{footnote}[10]\sphinxAtStartFootnote
“The Matrix” (1999), directed by Lana Wachowski and Lilly Wachowski
%
\end{footnote}

\sphinxAtStartPar
\sphinxincludegraphics[width=5.72224in,height=3.26985in]{{0d47b6834958c195dcaa3cae282c994f89983669}.png}

\sphinxAtStartPar
The developed AI will play a crucial role in the decision\sphinxhyphen{}making system
through a voting process. Its logic, endowed with impartiality and
integrity, will significantly contribute to all decision\sphinxhyphen{}making processes.

\sphinxAtStartPar
For AI to be fully realized, it requires integration into the real
world. Developers will assist AI by creating bridges—adapters for
interaction between AI and various application domains. Over time, AI
will be capable of creating these bridges independently. These bridges
facilitate the interaction of AI with specific application areas,
functioning as individual projects where AI will perform analytical
tasks, thereby aiding the project team.

\sphinxAtStartPar
During the operation of these projects, the IAAI token will be utilized.
A portion of the project’s profits will be allocated to repurchase the
token. The application areas can vary widely but will fundamentally be
those where AI can enhance the team’s efficiency. Each project will
essentially receive an ecosystem of tools: a DAO interface, IAAI as an
adapted AI, and a team of developers for implementation.

\sphinxAtStartPar
By integrating AI into the Blockchain and utilizing the IAAI token, we
aim to create a robust, decentralized decision\sphinxhyphen{}making system. This
system will not only improve efficiency and transparency but also ensure
that the AI’s contributions are valuable and incorruptible. Through this
innovative approach, we hope to pioneer new ways in which AI and
Blockchain can synergize for the betterment of various industries.


\chapter{The OODA Loop in effective digital decision making}
\label{\detokenize{index:the-ooda-loop-in-effective-digital-decision-making}}
\sphinxAtStartPar
The OODA Loop, created by John Boyd, is a model for decision\sphinxhyphen{}making
involving four interdependent tasks: Observe, Orient, Decide, and Act.

\sphinxAtStartPar
\sphinxincludegraphics[width=5.98819in,height=3.36111in]{{OODALoop}.png}

\sphinxAtStartPar
\sphinxstylestrong{Figure 1. John Boyd’s OODA loop}. (“John Boyd and John Warden:
Airpower’s Quest for Strategic Paralysis”)

\sphinxAtStartPar
OODA Loop is actually an activity network with rich information flows,
allowing for learning and adaptive decision\sphinxhyphen{}making. This loop is
applicable across various management levels and domains. The loop
involves selectively perceiving data, recognizing patterns to understand
situations, planning actions, and executing behaviors. The model
highlights the importance of feedback and learning, especially in the
context of big data and IoT, to improve situational awareness and
decision\sphinxhyphen{}making capabilities. Strategic project management can use the
OODA Loop to compare current capabilities with desired future outcomes,
guiding transformative actions or system improvements.

\sphinxAtStartPar
Modified version of the OODA Loop may help and can be used as an
Activity network in the separate Agents.


\chapter{Decision support based on BigData from Data Bridges}
\label{\detokenize{index:decision-support-based-on-bigdata-from-data-bridges}}
\sphinxAtStartPar
Big data and predictive analytics can enhance situational awareness by
providing new, relevant facts and patterns for decision\sphinxhyphen{}making. However,
to be effective, this data must be interpreted within a narrative AI
framework that gives meaning and builds trust in the system. Simply
using algorithms to gather and analyze data is insufficient for reliable
decision\sphinxhyphen{}making. Additionally, interpreting data can yield multiple
narratives, making it challenging to identify the correct one. To
address this, a secondary level of data collection should be proposed,
named here Data Bridge, focusing on creating models of situation types
that evolve with new data and insights. This dynamic approach ensures
continuous improvement and relevance, supporting effective
decision\sphinxhyphen{}making through an ongoing OODA loop for data interpretation.


\chapter{Autonomous Multi Systems in Decision Making}
\label{\detokenize{index:autonomous-multi-systems-in-decision-making}}
\sphinxAtStartPar
One of the ideas of the project is to make AI a key player in the
community, giving it the right and opportunity to analyze data and make
decisions.

\sphinxAtStartPar
The solution we propose on this problem layer is based on the idea of
autonomous digital decision making systems. Decisions are decentralized
and involve multiple entities working together, whether within a single
domain project or between multiple domain projects in collaborative
networks. Each participating entity in this system\sphinxhyphen{}of\sphinxhyphen{}systems (SoS)
has its own cryptographic identity, resources, communication
protocols and decision\sphinxhyphen{}making processes, supported by AI based decision
support systems (AI\sphinxhyphen{}DSS). %
\begin{footnote}[11]\sphinxAtStartFootnote
Henshaw, M., A Socio\sphinxhyphen{}Technical Perspective on SoSE, in Lecture Series
in Systems of Systems Engineering for NATO Defence Applications. 2014, NATO CSO.
%
\end{footnote}

\sphinxAtStartPar
Effective SoS decision\sphinxhyphen{}making requires a cooperative exchange of information and
commitments, ensuring systemic properties like availability are maintained, even
if individual systems fail. The architecture of a successful SoS must be dynamically
reconfigurable to preserve its functional integrity, mission
fulfillment, and management control. This robustness is achieved by:
\begin{enumerate}
\sphinxsetlistlabels{\arabic}{enumi}{enumii}{}{.}%
\item {} 
\sphinxAtStartPar
Designing the blockchain based decision function to handle incomplete
information temporarily.

\item {} 
\sphinxAtStartPar
Allowing the decision function to proactively request needed
information from contributing systems using the smart contracts as
the instructions.

\item {} 
\sphinxAtStartPar
Dynamically allocating OODA loop functions to resources, to ensure
capacity, availability, scalability, and other desirable properties.

\end{enumerate}

\sphinxAtStartPar
A SoS must be self\sphinxhyphen{}aware and capable of operational reconfiguration for
timely, reliable decision\sphinxhyphen{}making. The OODA loops must function at both
the SoS level and within individual participants, enabling optimized and
agile decision\sphinxhyphen{}making through a distributed paradigm.

\sphinxAtStartPar
The idea of the project is to create a protocol and infrastructure
that allows the deployment of a domain specific DAO, the key player of
which is IA. Domains can be anything, it completely depends on the tasks
of a particular project. Here are some examples: creating a real estate
project involved in financial analysis and subsequent investments in
real estate. IA’s task is to analyze data about real estate objects
coming through the bridges and create investment advice. The task of the
project developers is to prepare and configure data bridges through
third\sphinxhyphen{}party APIs. The task of project managers / users is to set up the
project by creating the roles and rights of the DAO project
participants, assign weights and roles to each. Subsequently, prepare
halls for voting and events. At all these stages, the IA can act as an
equal member of the community.

\sphinxAtStartPar
\sphinxincludegraphics[width=5.98819in,height=5.66667in]{{IAAIArchitecture}.png}

\sphinxAtStartPar
\sphinxstylestrong{Figure 2. IAAI Architecture}


\chapter{IAAI Network}
\label{\detokenize{index:iaai-network}}
\sphinxAtStartPar
After the MEME project phase IAAI will be transformed into the coin on
its own ETH2 like EVM Blockchain network, based on PoS. This step will
allow the development of new ERC standards that are more appropriate
to the IA target tasks and needs like communication between core IAAI
and IA agents, communication between IA and DAO participants, storing of
the AI retrained data on the blockchain etc. General network architecture will looks like:

\sphinxAtStartPar
\sphinxincludegraphics[width=5.98819in,height=5.66667in]{{IAAIBlockchainArchitecture}.png}

\sphinxAtStartPar
The IAAI infrastructure consists of two parts: the onchain layer and the offchain layer.
The first is a blockchain execution layer in a virtual machine and is deployed in a blockchain network.
The offchain layer is deployed for each project in an isolated cloud environment.
This is done to achieve greater computational capabilities in development as resource\sphinxhyphen{}intensive tasks will be executed in this environment.
\begin{itemize}
\item {} 
\sphinxAtStartPar
It hosts the IAAI agent, which performs its autonomous and project\sphinxhyphen{}specific role;

\item {} 
\sphinxAtStartPar
Project APPs \sphinxhyphen{} numerous applications within the project;

\item {} 
\sphinxAtStartPar
Project API \sphinxhyphen{} an API for connecting the external world with project endpoints;

\item {} 
\sphinxAtStartPar
Project Token \sphinxhyphen{} stores a reference to the project tokens deployed on the onchain;

\item {} 
\sphinxAtStartPar
Project Storage \sphinxhyphen{} the project’s storage for all necessary data.

\end{itemize}

\sphinxAtStartPar
This architecture allows for the deployment and organization of collaborative work between the IAAI Agent and project participants.
By endowing the project with tokens, we can embed an economic component into the project.
By providing project participants with onchain identity, we can turn the project into a decentralized organization.


\chapter{\sphinxstylestrong{Conclusion}}
\label{\detokenize{index:conclusion}}
\sphinxAtStartPar
The introduction of a fully controlled blockchain network with
artificial intelligence can increase project power, significantly reduce
human errors in decision\sphinxhyphen{}making processes, speed up and delegate human
routine, without losing control, but to the required extent, giving it
to AI. By eliminating the need for trusted human intermediaries, the
system ensures more reliable and efficient outcomes. Our approach
leverages an AI blockchain network and sophisticated AI\sphinxhyphen{}to\sphinxhyphen{}human
communication protocols to create an autonomous decision support system
(AI\sphinxhyphen{}DSS). Each decentralized autonomous organization (DAO) within the
network operates independently, using specialized AI to facilitate
informed decision\sphinxhyphen{}making and provide a seamless interface for human
interaction. Furthermore, the development of a digital decision
analytics system is essential. This system will incorporate Situated
Logic to generate meaningful narratives from analytics results and
employ Channel Theory to enhance situational awareness, ensuring
comprehensive and contextually relevant decision support. This
integrated approach promises to revolutionize decision\sphinxhyphen{}making by
combining the strengths of AI and blockchain technology with advanced
analytical techniques.



\renewcommand{\indexname}{Index}
\printindex
\end{document}